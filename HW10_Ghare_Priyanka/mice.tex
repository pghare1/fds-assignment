
\documentclass[11pt]{article} 

\usepackage[utf8]{inputenc} 

\usepackage{geometry} 
\geometry{a4paper} 

\usepackage{graphicx} 



%%% PACKAGES
\usepackage{booktabs} 
\usepackage{array} 
\usepackage{paralist} 
\usepackage{verbatim} 
\usepackage{subfig} 



\usepackage{fancyhdr} 
\pagestyle{fancy} 
\renewcommand{\headrulewidth}{0pt} 
\lhead{}\chead{}\rhead{}
\lfoot{}\cfoot{\thepage}\rfoot{}


\usepackage{sectsty}
\allsectionsfont{\sffamily\mdseries\upshape}



\usepackage[nottoc,notlof,notlot]{tocbibind} 
\usepackage[titles,subfigure]{tocloft}
\renewcommand{\cftsecfont}{\rmfamily\mdseries\upshape}
\renewcommand{\cftsecpagefont}{\rmfamily\mdseries\upshape}



\title{Homework 10}
\author{Priyanka Mahendra Ghare}
\date{November 6, 2022}

\begin{document}
\maketitle

\section*{2. X2 testing for independence between categorical variables}

\begin{verbatim}
Possible Gene Types:['J', 'R', 'K']

Contingency table:
       No Cancer  Has Cancer
Gene
J             93          37  130
K             34           5   39
R             20           1   21
Total        147          43  190

Conditional proportions table:
      No Cancer Has Cancer row_marginal_per
Gene
J         71.5%      28.5%            68.4%
K         87.2%      12.8%            20.5%
R         95.2%       4.8%            11.1%
Total     77.4%      22.6%           100.0%

Expected counts table:
        No Cancer Has Cancer  Total
Gene
J      100.578947  29.421053  130.0
K       30.173684   8.826316   39.0
R       16.247368   4.752632   21.0
Total       77.4%      22.6%

 X^2 = 8.50
Degree of freedom = 2
P-value = 0.0143
\end{verbatim}
We reject the null hypothesis that the gene and cancer are independent because the p-value is less than 0.05. Therefore, it would seem extremely unlikely that the gene and cancer are independent.
\end{document}
